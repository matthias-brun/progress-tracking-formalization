\documentclass[11pt,a4paper]{article}
\usepackage{isabelle,isabellesym}

\usepackage{cite}
\usepackage{amssymb}

\usepackage{pdfsetup}

\urlstyle{rm}
\isabellestyle{it}


\begin{document}

\title{Formalization of Timely Dataflow's\\ Progress Tracking Protocol}
\author{Matthias Brun and S\'ara Decova and Andrea Lattuada and Dmitriy Traytel}

\maketitle

\begin{abstract}
Large-scale stream processing systems often follow the dataflow paradigm, which enforces a program
structure that exposes a high degree of parallelism. The Timely Dataflow distributed system supports
expressive cyclic dataflows for which it offers low-latency data- and pipeline-parallel stream processing.
To achieve high expressiveness and performance, Timely Dataflow uses an intricate distributed protocol
for tracking the computation’s progress. We formalize this progress tracking protocol and verify its safety.
Our formalization is described in detail in the forthcoming ITP'21 paper~\cite{BrunDLT-ITP21}.
\end{abstract}

\tableofcontents

% sane default for proof documents
\parindent 0pt\parskip 0.5ex

\section{Introduction}

The dataflow programming model represents a program as a directed graph of interconnected operators
that perform per-tuple data transformations. A message (an incoming datum) arrives at an input (a
root of the dataflow) and flows along the graph's edges into operators. Each operator takes the
message, processes it, and emits any resulting derived messages.

In a dataflow system, all messages are associated with a timestamp, and operator instances need to
know up-to-date (timestamp) \textit{frontiers}---lower bounds on what timestamps may still appear as
their inputs. When informed that all data for a range of timestamps has been delivered, an operator
instance can complete the computation on input data for that range of timestamps, produce the
resulting output, and retire those timestamps.

A \textit{progress tracking mechanism} is a core component of the dataflow system. It receives
information on outstanding timestamps from operator instances, exchanges this information with other
system workers (cores, nodes) and disseminates up-to-date approximations of the frontiers to all
operator instances. This AFP entry formally models and proves the safety of the progress
tracking protocol of \textit{Timely
Dataflow}~\cite{DBLP:conf/sosp/MurrayMIIBA13,URL:timely-dataflow}, a dataflow programming
model and a state-of-the-art streaming, data-parallel, distributed data processor.
Specifically, we
prove that the progress tracking protocol computes frontiers that always constitute safe lower bounds on what
timestamps may still appear on the operator inputs.
The formalization is described in detail in the forthcoming ITP'21 paper~\cite{BrunDLT-ITP21}.

The ITP paper~\cite{BrunDLT-ITP21} closely follows this formalization's structure. In particular, the paper's presentation
is split into four main sections each of which is present in the formalization (each in a separate
theory file):
\begin{center}
\begin{tabular}{@{}p{0.27\textwidth}p{0.215\textwidth}p{0.168\textwidth}p{0.243\textwidth}@{}}
Algorithm/protocol&Section in this proof document&Section in \cite{BrunDLT-ITP21}&Theory file\\\hline
Abadi et al.~\cite{DBLP:conf/forte/AbadiMMR13}'s clocks protocol&Section~\ref{sec:clocks}&Section 3&Exchange\_Abadi\\
Exchange protocol&Section~\ref{sec:exchange}&Section 4&Exchange\\
Local propagation algorithm&Section~\ref{sec:propagate}&Section 5&Propagate\\
Combined protocol&Section~\ref{sec:combined}&Section 6&Combined
\end{tabular}
\end{center}

\input{session}

% optional bibliography
\bibliographystyle{abbrv}
\bibliography{root}

\end{document}

%%% Local Variables:
%%% mode: latex
%%% TeX-master: t
%%% End:
